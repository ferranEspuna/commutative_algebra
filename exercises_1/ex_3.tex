
% PROBLEM 3
\begin{problem}
    Let $A$ be a ring.
    We define the Jacobson radical $\mathcal{J}(A)$ as the intersection of all maximal ideals of $A$.
    Prove:
    \begin{enumerate}[label=(\theproblem.\arabic*),ref=\theproblem.\arabic*]

        % PROBLEM 3.1
        \item \label{itm:jacobson_element_to_invertible}
        $x \in \mathcal{J}(A) \Leftrightarrow 1-xy$ is invertible for all $y \in A$.
            \begin{sol}
                \begin{proof}
                We will prove the two implications separately:
                    \begin{itemize}
                        \item[$(\Leftarrow)$]
                        Suppose $x \notin \mathcal{J}(A)$.
                        This means that there exists $\mathfrak{m} \in \text{Max}\,A$ such that $x \notin \mathfrak{m}$.
                        Since $\mathfrak{m}$ is a maximal ideal, $(x) + \mathfrak{m} = A$.
                        Then, there exist $y' \in A$ and $m' \in \mathfrak{m}$ such that $x y' + m' = 1 \in A$.
                        But this means that $1 - x y' = m' \in \mathfrak{m}$, which invertible.
                        Contradiction with $\mathfrak{m}$ being an ideal.
                        Thus, $x \mathcal{J}(A)$.
                        \item[$(\Rightarrow)$]
                        Suppose that $\exists y \in A$ such that $1-x y$ is not a unit in $A$.
                        Notice that:
                        \begin{itemize}
                            \item There exists a maximal ideal $\mathfrak{M}$ such that $1 - x y \in \mathfrak{m}$.
                            \item Since $x \in \mathfrak{m} \subseteq \mathcal{J}(A)$, we have that $x y \in \mathfrak{m}$.
                        \end{itemize}
                        Finally,  $1 - x y + x y = 1 \in \mathfrak{m}$, in contradiction with $\mathfrak{m}$ being an ideal.
                        Thus, $\forall y \in A$ we have that $1 - x y$ is a unit in $A$.
                    \end{itemize}
                \end{proof}
            \end{sol}

        % PROBLEM 3.2
        \item $\mathcal{J}(A) = A \setminus A* \Leftrightarrow A$ has only one maximal ideal.
            \begin{sol}
                \begin{proof}
                    This comes directly from the fact that an ideal $I \subset A$ cannot contain a unit $x$, else $x x^{-1} = 1 \in I$ and thus $I = A$.
                    \begin{gather*}
                        \mathcal{J}(A) = A \setminus A^* \Leftrightarrow \forall \mathfrak{m} \in \text{Max}\,A, \mathfrak{m} \supseteq \{\text{non-units of }A\} = A \setminus A^* \Leftrightarrow\\
                        \Leftrightarrow \forall \mathfrak{m} \in \text{Max}\,A, \mathfrak{m} = \{\text{non-units of }A\} \Leftrightarrow (A, \mathfrak{m}) \text{ local}\\
                    \end{gather*}
                \end{proof}
            \end{sol}

        % PROBLEM 3.3
        \item \label{itm:nilradical_equals_jacobson_in_polynomial_ring}
        $\mathcal{J}(A[T]) = \mathcal{N}(A[T])$.
            \begin{sol}
                \begin{proof}
                We will prove the two inclusions separately:
                    \begin{itemize}
                        \item[$(\supseteq)$]
                        \[ \mathcal{N}(A[T]) = \bigcap \text{Spec}\,A[T] \subseteq \bigcap \text{Max}\,A[T] = \mathcal{J}(A[T]) \]
                        The first equality follows~\ref{itm:nilradical_is_intersection_of_all_primes}, while the inclusion is a consequence of $\text{Max}\,A[T] \subseteq \text{Spec}\,A[T]$.
                        \item[$(\subseteq)$]
                        Take $f = a_0 + a_1 T + \cdots + a_n T^n \in \mathcal{J}(A[T])$.
                        Using ~\ref{itm:jacobson_element_to_invertible} with $y = T$ we know that $1 - f T = 1 - a_0 T - a_1 T^2 \cdots - a_n T^{n + 1}$ is invertible in $A[t]$.
                        But then, from ~\ref{itm:coefficients_nilpotent} we get that $-a_0, \cdots, -a_n$ nilponents, and finally $a_0, \cdots a_n$ are nilponents.
                        We conclude from~\ref{itm:nilponent_implies_nilponent_coefficients} we get that $f$ is nilponent in $A[T]$
                    \end{itemize}
                \end{proof}
            \end{sol}
    \end{enumerate}
\end{problem}
