\begin{problem}
    Let $A$ be a ring and let $I$ be an ideal of $A$.
    Prove:
    \begin{enumerate}[label=(\theproblem.\arabic*),ref=\theproblem.\arabic*]

        % PROBLEM 13.1
        \item If $\mathfrak{p}$ is a minimal prime of $I$, then $\mathfrak{p}[T]$ is a minimal prime of $I[T]$, and any minimal prime of $I[T]$ is of this form.
        \begin{sol}

            \begin{proof}

                \begin{claim}
                    If $\mathfrak{q} \supset I[T]$ is a prime ideal of $A[T]$, then exists $\mathfrak{p}$ prime ideal of $A$ such that
                    $\mathfrak{q} \supset \mathfrak{p}[T] \supset I[T]$.
                     \begin{proof}
                        Consider the usual inclusion $A \subseteq A[T]$.
                        Then,
                        $\mathfrak{p} \coloneqq \mathfrak{q}^{c}$ is a prime ideal of $A$.
                        Furthermore,
                        \[I = I[T]^c \subset \mathfrak{q}^c = \mathfrak{p}\]
                        and
                        \[I[T] = I^e \subset \mathfrak{q}^{ce} \subset \mathfrak{q}\]
                        with $\mathfrak{q}^{ce}=\mathfrak{p}^e = \mathfrak{p}[T]$.
                     \end{proof}
                \end{claim}

                \begin{claim}
                    If $\mathfrak{p}$ is an ideal of $A$, $\mathfrak{p}$ prime in $A \Leftrightarrow \mathfrak{p}[T]$ prime in $A[T]$ .
                    \begin{proof}
                        $\mathfrak{p} \text{ prime } \Leftrightarrow
                        A / \mathfrak{p} \text{ integral domain }
                        \Leftrightarrow A[T]/\mathfrak{p}[T] \cong (A / \mathfrak{p})[T] \text{ integral domain }
                        \Leftrightarrow \mathfrak{p}[T] \text{ prime.}$
                    \end{proof}
                \end{claim}



                Suppose $\mathfrak{q} \subseteq A[T]$ is a minimal prime of $I[T]$.
                Then, by the first claim, we have some prime ideal $\mathfrak{p} \subset A$ such that
                $\mathfrak{q} \supset \mathfrak{p}[T] \supset I[T]$,
                By the second claim, $\mathfrak{p}[T]$ is prime.
                Therefore, $\mathfrak{p}[T] = \mathfrak{q}$.

                Now, let $\mathfrak{p}$ be a minimal prime of $I$.
                $\mathfrak{p}[T]$ must contain a minimal prime of $I[T]$, say, $\mathfrak{p}'[T]$
                Then, we take the contraction $I \subseteq \mathfrak{p}' \subseteq \mathfrak{p}$
                and by definition of $\mathfrak{p}$ being a minimal of $I$,
                we conclude that $\mathfrak{p}' = \mathfrak{p}$ and $\mathfrak{p}'[T] = \mathfrak{p}[T]$.
                Thus, $\mathfrak{p}[T]$ is a minimal prime of $I[T]$.

            \end{proof}
        \end{sol}

        % PROBLEM 13.2
        \item\label{itm:p_primary_implies_pt_primary} If $\mathfrak{q}$ is $\mathfrak{p}$-primary ideal of $A$, then $\mathfrak{q}[T]$ is a $\mathfrak{p}[T]$-primary ideal of $A[T]$.
        \begin{sol}

            \begin{proof}

            Note that, for $ f \in A[T], \, f \in rad(\mathfrak{q}[T]) \Leftrightarrow f^n \in \mathfrak{q}[T] \text{ for some } n > 0
            \Leftrightarrow f \text{ nilpotent in } A[T] / \mathfrak{q}[T] \cong (A/\mathfrak{q})[T]$.
            By~\ref{itm:nilponent_implies_nilponent_coefficients}, this happens precisely when
            the coefficients of $f$ are nilpotent in $A/\mathfrak{q}$.
            That is, when they belong to $\text{rad}(\mathfrak{q})$.
            We then have the equality $\text{rad}(q[T]) = \text{rad}(q)[T]$.

            Furthermore, we have shown that $\mathfrak{p}[T]$ is prime in $A[T]$ exactly when $\mathfrak{p}$ is prime in $A$.
            The same is true for the radicals, so we are done.
            \end{proof}


        \end{sol}

        % PROBLEM 13.3
        \item If $I = \mathfrak{q}_1 \inter \cdots \inter \mathfrak{q}_n$ is a minimal primary decomposition of $I$, then $I[T] = \mathfrak{q}_1[T] \inter \cdots \inter \mathfrak{q}_n[T]$ is a minimal primary decomposition of $I[T]$.
        \begin{sol}
            First of all we see that, in fact,
            \begin{gather*}
                I[T] = \Bigl\{ \text{polynomials with coefficients in } I \Bigr\} =\\
                = \Bigl\{ \text{polynomials with coefficients in } \mathfrak{q}_1[T] \text{, } \mathfrak{q}_2[T] \text{, ... and } \mathfrak{q}_n[T] \Bigr\} =\\
                = \bigcap_{i=1}^n \Bigl\{ \text{polynomials with coefficients in } \mathfrak{q}_i[T] \Bigr\} = \mathfrak{q}_1[T] \inter \cdots \inter \mathfrak{q}_n[T]\\
            \end{gather*}
            Now, we know that for all $i \in \{ 1, \cdots, n \}$, $\mathfrak{q}_i$ is $\mathfrak{p}_i$-primary for some prime $\mathfrak{p}_i$.
            Following~\ref{itm:p_primary_implies_pt_primary}, for all $i \in \{ 1, \cdots, n \}$, $\mathfrak{q}_i[T]$ is $\mathfrak{p}_i[T]$-primary and thus, $I[T] = \mathfrak{q}_1[T] \inter \cdots \inter \mathfrak{q}_n[T]$ is a primary decomposition of $I[T]$.
            We just need to prove that it is minimal:
            \begin{itemize}
                \item
                Suppose that $rad(\mathfrak{q}_i[T]) = \mathfrak{p}_i[T] = \mathfrak{p}_j[T] = rad(\mathfrak{q}_j[T])$ for some $i \neq j$.
                Then, we can take the contraction $\mathfrak{p}_i = \mathfrak{p}_i[T]^c = \mathfrak{p}_j[T]^c = \mathfrak{p}_j$, which contradicts $I = \mathfrak{q}_1 \inter \cdots \inter \mathfrak{q}_n$ being a minimal primary decomposition of $I$.
                Thus, the $\mathfrak{p}_1[T], \cdots, \mathfrak{p}_n[T]$ are all different.
                \item
                Take $i \in \{ 1, \cdots, n \}$ and suppose that $\bigcap_{j \neq i} \mathfrak{q}_j[T] \subset \mathfrak{q}_i$.
                Then
                \begin{gather*}
                    \Bigl( \bigcap_{j \neq i} \mathfrak{q}_j \Bigr)[T] = \Bigl\{ \text{polynomials with coefficients in all } \mathfrak{q}_j \text{ with } i \neq j \Bigr\} =\\
                    = \bigcap_{j \neq i} \Bigl\{ \text{polynomials with coefficients in } \mathfrak{q}_j \Bigr\} = \bigcap_{j \neq i} \mathfrak{q}_j[T] \subset \mathfrak{q}_i\\
                \end{gather*}
                and we have that $\bigcap_{j \neq i} \mathfrak{q}_j \subset \mathfrak{q}_i$, in contradiction with $I = \mathfrak{q}_1 \inter \cdots \inter \mathfrak{q}_n$ being a minimal primary decomposition of $I$.
                Thus, $\bigcap_{j \neq i} \mathfrak{q}_j[T] \nsubseteq \mathfrak{q}_i, \forall i \in \{ 1, \cdots, n \}$.
            \end{itemize}
            We conclude that $I[T] = \mathfrak{q}_1[T] \inter \cdots \inter \mathfrak{q}_n[T]$ is a minimal primary decomposition of $I[T]$.
        \end{sol}

    \end{enumerate}
\end{problem}