\begin{problem}\label{itm:integers_polynomial_ring_spectrum}
Describe $\text{spec}(\mathbb{Z}[T])$
\begin{sol}

    Because $\mathbb{Z}$ is a PID, we know that
    $\text{Spec}\,\mathbb{Z} = \{0\} \cup \{(p) \mid p \text{ prime}\}$.
    We can describe the spectrum of $\mathbb{Z}[T]$ in terms of the spectrum of $\mathbb{Z}$,
    via the natural inclusion $\phi:\mathbb{Z} \to \mathbb{Z}[T]$.
    This is because, as we have discussed in class,

    \[
        f^{*-1}(\mathfrak{p})
        \coloneqq
        \{\mathfrak{q} \in \text{Spec} \, \mathbb{Z}[T] \mid \mathfrak{q}^c = \mathfrak{p}\}
        \cong \text{Spec} \, \phi(\mathbb{Z}[T] \setminus \mathfrak{p})^{-1}( \mathbb{Z}[T] / \mathfrak{p}^e)
    \]
    So, to get an exhaustive list of the prime ideals of $\mathbb{Z}[T]$, we just have to take the
    ideals in this new ring for each prime ideal $\mathfrak{p}$ of $\mathbb{Z}$, and then
    contract them back to $\mathbb{Z}[T]$.
    We distinguish two cases:

    \begin{itemize}
        \item $\mathfrak{p} = (p)$ where $p$ is prime:
        The ring to study is $(\mathbb{Z} \setminus p \mathbb{Z})^{-1} (\mathbb{Z}[T]/p\mathbb{Z}[T]) \cong \linebreak
        (\mathbb{Z}/p\mathbb{Z} \setminus \bar{0} \mathbb{Z})^{-1} (\mathbb{Z}/p\mathbb{Z})[T]$.
        However, all non-zero elements of $\mathbb{Z}/p\mathbb{Z}$ are units, so this is just
        $(\mathbb{Z}/p\mathbb{Z})[T]$.
        This is a PID because $\mathbb{Z}/p\mathbb{Z}$ is a field.
        Therefore, the only prime ideals are $(0)$ and $(\bar{f})$ where $f$ is irreducible modulo $p$.

        \begin{itemize}
            \item $(\bar{0})$ contracts to $(p)$.
            \item $(\bar{f})$ with f irreducible mod p: Again, because we are working in over a field,
            we can take  $f$ to be monic by multiplying by a unit.
            the preimage of an element $(\bar{f}\bar{g})$ of $(f)$ can be any element of the
            form $f g + h p$ where $h \in \mathbb{Z}[T]$.
            These are precisely the elements of the ideal
            $(f) + (p) = (f, p)$.

        \end{itemize}

        \item  $\mathfrak{p} = (0)$: The ring to study is
        $(\mathbb{Z} \setminus \{0\})^{-1} \mathbb{Z}[T] = \mathbb{Q}[T]$.
        Since $\mathbb{Q}$ is a field, the only prime ideals are $(0)$ and $(f)$ where $f$ is irreducible in $\mathbb{Q}[T]$:
        \begin{itemize}
            \item $(0)$ contracts to $(0)$.
            \item $(f)$ with $f$ irreducible in $\mathbb{Q}[T]$:
            To contract these ideals back to $\mathbb{Z}[T]$,
            is to just intersect them with $\mathbb{Z}[T]$.
            Suppose we have an irreducible polynomial $f \in \mathbb{Q}[T]$.
            As far as the prime ideal is concerned, we can assume $f$
            has coefficients in $\mathbb{Z}$, and is primitive.
            the elements of $(f)$ are of the form $\frac{\lambda}{\mu}fg$
            where $\frac{\lambda}{\mu} \in \mathbb{Q}$ and $g \in \mathbb{Z}[T]$ primitive
            (we can absorb any common factors of $g$ into $\frac{\lambda}{\mu}$).
            Intersecting with $\mathbb{Z}[T]$, because $fg$ is primitive (This is Gauss' Lemma),
            this means that $\frac{\lambda}{\mu} \in \mathbb{Z}$,
            so $(f) \cap \mathbb{Z}[T] = (f)_{\mathbb{Z[T]}}$.
        \end{itemize}
    \end{itemize}

    Putting all of this together, we get that the prime ideals of $\mathbb{Z}[T]$ are:
    \begin{itemize}
        \item $(p)$ where $p$ is prime.
        \item $(f, p)$ where $f$ is irreducible mod $p$ and $p$ is prime.
        \item $(0)$.
        \item $(f)$ where $f$ is irreducible in $\mathbb{Q}[T]$ and primitive.
    \end{itemize}
    For a quick proof of Gauss's Lemma, we can use the fact that for $p \in \mathbb{Z}$ prime, $p\mathbb{Z}[T]$ is prime in $\mathbb{Z}[T]$
    (either by this exercise or by the fact that $\mathbb{Z}[T]/(p) \cong \mathbb{F}_p[T]$ is a domain).
    Therefore if a prime factor divides the coefficients of $fg$, it must divide either $f$ or $g$.
    But if $f, g$ are primitive, this is impossible.



\end{sol}
\end{problem}