\documentclass[11pt]{article}
\usepackage[margin=3cm]{geometry}
\usepackage{amsmath}
\usepackage{amsthm}
\usepackage{mathtools}
\usepackage{amsfonts}
\usepackage{amssymb}
\usepackage{tikz-cd}
\usepackage{wasysym}
\usepackage{stmaryrd}
\usepackage{xfrac}
\usepackage{enumitem}
%These are my Commutative Algebra Teory notes from the course given by Prof. Josep Alvarez Montaner at UPC in 2023.

% Title
\title{Commutative Algebra. Exam Theory}
\author{Ferran Espuña, Severino Da Dalt}

\newtheorem{theorem}{Theorem}[section]
\newtheorem{corollary}{Corollary}[theorem]
\newtheorem{lemma}[theorem]{Lemma}
\newtheorem{prop}[theorem]{Proposition}
\newtheorem{exercise}[theorem]{Exercise}
\newtheorem*{claim}{Claim}


\theoremstyle{definition}
\newtheorem{defn}[theorem]{Definition}
\newtheorem{rk}[theorem]{Remark}
\newtheorem{example}[theorem]{Example}


% Commands
\newcommand{\notimplies}{\;\not\!\!\!\implies}

% Document
\begin{document}

    \maketitle

    \section{Chain Conditions}\label{sec:chain-conditions}
    \begin{theorem}
    Let A be a ring.
    The following are equivalent:
    \begin{enumerate}

        \item \label{itm:fg}
        Every ideal of $A$ is finitely generated.

        \item \label{itm:acc}
        \emph{\textbf{Ascending chain condition}}:
        every chain of ideals $I_1 \subseteq I_2 \subseteq \cdots \subseteq A$
        is stationary, i.e.
        there exists $n$ such that $I_n = I_{m} = \forall m \geq n$.

        \item \label{itm:max}
        Every non-empty set of ideals of $A$ has a maximal element.

    \end{enumerate}

    \begin{proof}
        We will prove $ \eqref{itm:fg} \implies \eqref{itm:acc} \implies \eqref{itm:max} \implies \eqref{itm:fg}$.
        \begin{itemize}[align=left]

            \item[$\eqref{itm:fg} \Rightarrow \eqref{itm:acc}$:]
            Consider $I \coloneqq \bigcup_{i \in \mathbb{N}} I_n$.
            It is a finitely generated ideal, say $I = \left<a_1, \dots, a_n\right>$.
            Let $a_j \in I_{i_j}$ for each $j \leq n$ and $k = \max_j i_j$.
            Then, $I = I_k$ so the chain is stationary.

            \item[$\eqref{itm:acc} \Rightarrow \eqref{itm:max}$:]
            Let $\mathcal{F}$ be a non-empty set of ideals of $A$.
            Suppose that the condition is false.
            Pick $I_1 \in \mathcal{F}$.
            Because it is not maximal, there exists $I_2 \in \mathcal{F}$ such that $I_1 \subsetneq I_2$.
            By iterating this process, we get a chain $I_1 \subsetneq I_2 \subsetneq \cdots$ which is not stationary.

            \item[$\eqref{itm:max} \Rightarrow \eqref{itm:fg}$:]
            Let $I$ be an ideal of $A$.
            consider the set of finitely generated ideals of $A$ contained in $I$.
            It is non-empty because $0 \in I$.
            By hypothesis, it has a maximal element, say $J$.
            We claim that $J = I$.
            Otherwise,
            there exists $a \in I \setminus J$,
            but then $J + (a)$ is a finitely generated ideal of $A$ contained in
            $I$ which strictly contains $J$.
            in contradiction with the maximality of $J$.
        \end{itemize}
    \end{proof}
\end{theorem}

\begin{defn}
    A ring $A$ satisfying the equivalent conditions of the previous theorem is called \emph{Noetherian}.
\end{defn}

\begin{theorem}
    Let $A$ be a ring.
    The following are equivalent:
    \begin{enumerate}
        \item \label{itm:dcc}
        \emph{\textbf{Descending chain condition}}:
        Every of ideals $A \supseteq I_1 \supseteq I_2 \supseteq \cdots$,
        is stationary, i.e.
        there exists $n$ such that $I_n = I_{m} = \forall m \geq n$.

        \item \label{itm:min}
        Every non-empty set of ideals of $A$ has a minimal element.
    \end{enumerate}

    \begin{proof}
        We will prove $ \eqref{itm:dcc} \implies \eqref{itm:min} \implies \eqref{itm:dcc}$.
        \begin{itemize}[align=left]

            \item[$\eqref{itm:dcc} \Rightarrow \eqref{itm:min}$:]
            Just replace ``maximal'' with ``minimal'' in the proof of $\eqref{itm:acc} \Rightarrow \eqref{itm:max}$.

            \item[$\eqref{itm:min} \Rightarrow \eqref{itm:dcc}$:]
            Let $I_1 \supseteq I_2 \supseteq \cdots$ be a chain of ideals of $A$.
            Then the set $\mathcal{F} \coloneqq \{I_n\}_{n \in \mathbb{N}}$ is non-empty.
            By hypothesis, it has a minimal element, say $I_k$.
            Then, $I_k = I_n$ for all $n \geq k$, each inclusion is shown by the chain condition or by minimality of $I_k$.

        \end{itemize}
    \end{proof}
\end{theorem}

\begin{defn}
    A ring $A$ satisfying the equivalent conditions of the previous theorem is called \emph{Artinian}.
\end{defn}

\begin{rk}
    These two conditions are preserved by taking quotients and rings of fractions.
    This is because we can map all ideals of $A/I$ and $S^{-1} A$ to ideals of $A$
    in an injective way that preserves inclusions (lifting and intersection, respectively).
\end{rk}

\begin{theorem}[Hilbert's basis theorem]
    $A$ Noetherian $\implies A[x]$ Noetherian.
    \begin{proof}
        Suppose that $A$ is Noetherian and let $I \subseteq A[x]$ be an ideal.
        Suppose that it is not finitely generated.
        Let $f_1 \in I \setminus 0,
        f_2 \in I \setminus (f_1),
        f_3 \in I \setminus (f_1, f_2),
        \dots$.
        We may suppose that we have chosen the $f_i$ such that $\deg f_i$ is minimal at each step.
        This means that if the leading term of $f_i$ is $a_i x^{d_i}$,
        $d_1 \leq d_2 \leq d_3 \cdots$.
        Because $A$ is Noetherian, the chain
        $(a_1) \subseteq (a_1, a_2) \subseteq \cdots \subseteq A$
        stabilizes, say, at step k.
        Therefore, we can express, for some $b_i$s $\in A$,
        $a_{k+1} = a_1 b_1 + \cdots + a_{k} b_{k}$.
        Now, consider the polynomial
        \[
            g = f_{k+1} - ( b_1 x^{d_{k+1} - d_1} f_1 + \cdots + b_k x^{d_{k+1} - d_k} f_k)
        \]
        Each of the terms in the sum has degree at most $d_{k+1}$,
        but by construction the degree $k+1$ coefficient of $g$ is zero.
        However, $g \in (f_1, \dots, f_k, f_{k+1}) \setminus (f_1, \dots, f_k)$.
        Otherwise,
        \[
        f_{k+1} = g + b_1 x^{d_{k+1} - d_1} f_1 +
        \cdots + b_k x^{d_{k+1} - d_k} f_k \in (f_1, \dots, f_k)
        \]
        This contradicts the minimality of $d_{k+1}$.


    \end{proof}
\end{theorem}

\begin{rk}
    This in turn implies that $A[x_1, \dots, x_n]$ is Noetherian, by induction on $n$.
\end{rk}

\begin{rk}
    The same is not true for Artinian rings.
    For example, if $\mathbb{K}$ is a field, $\mathbb{K}[x]$ is not Artinian.
    Indeed, the chain $(x) \supsetneq (x^2) \supsetneq (x^3) \supsetneq \cdots$ is not stationary.
\end{rk}

    \section{Primary Decomposition}\label{sec:primary-decomposition}
    \begin{defn}
        Let $\mathfrak{q} \in A$ be a proper ideal.
        We say that $\mathfrak{q}$ is \emph{primary} if fot all $a, b \in A$
        \[
            ab \in \mathfrak{q}, a \notin \mathfrak{q} \Longrightarrow b^n \in \mathfrak{q} \text{ for some } n > 0
        \]
\end{defn}

\begin{rk}
    We have:
    \begin{itemize}
        \item $\mathfrak{p}$ prime ideal $\Longrightarrow \mathfrak{p}$ primary ideal.
        \item $\mathfrak{p}$ primary ideal $\notimplies \mathfrak{p}$ prime ideal.
        \item Let $f: A \longrightarrow B$ be a ring homomorphism.
        Then,
        \[
            \mathfrak{q} \subseteq B \text{ primary ideal } \Longrightarrow \mathfrak{q}^c \subseteq A \text{ primary ideal}
        \]
    \end{itemize}
\end{rk}

\begin{prop}
    $\mathfrak{q} \text{ primary ideal } \Longrightarrow rad(\mathfrak{q}) \text{ prime ideal }$
\end{prop}
    \begin{proof}

        Let $ab \in rad(\mathfrak{q})$.
        Suppose $a \notin rad(\mathfrak{q})$.
        In particular, $a \notin \mathfrak{q}$.
        Then, $b^n \in \mathfrak{q}$ for some $n > 0$.
        Therefore, $b \in rad(\mathfrak{q})$.

    \end{proof}

\begin{defn}
    In the conditions of the previous proposition, we say that $\mathfrak{q}$ is $\mathfrak{p}$-primary.
\end{defn}


\begin{prop}
    Let $\mathfrak{q} \subseteq A$ be an ideal s.t. $rad(\mathfrak{q}) = \mathfrak{m}$ is a maximal ideal.
    Then $\mathfrak{q}$ is primary.

    \begin{proof}
        Let $ab \in \mathfrak{q}$.
        Consider the inclusions $\mathfrak{m} \subseteq \mathfrak{m} + (b) \subseteq A$.
        Because $\mathfrak{m}$ is maximal, we must have equality in (exactly) one of the two:

        \begin{itemize}

            \item If the second inclusion is an equality, we have $1 = x + by$, for some $x \in \mathfrak{m}$ and $y \in A$.
            Then, $a = ax + aby \Rightarrow a(1-x) = aby \in \mathfrak{q}$.
            However, reducing modulo $\mathfrak{q}$, we see $\bar{a}\overline{(1-x)} = \bar{0}$
            and $\overline{x} \in \mathcal{N}(A/\mathfrak{q}) \Rightarrow \overline{1-x}$ is a unit.
            Therefore. $\bar{a} = \bar{0}$, which means $a \in \mathfrak{q}$.

            \item If the first inclusion is an equality, we have
            $b \in \mathfrak{m} = rad(\mathfrak{q})
            \Rightarrow b^n \in \mathfrak{q}$ for some $n > 0$.

        \end{itemize}
    \end{proof}

\end{prop}

\begin{defn}
    Let $A$ be a ring and $I \subseteq A$ an ideal.
    A \emph{primary decomposition} of $I$ is a representation of $I$ as an intersection of primary ideals,
    say, $I = \mathfrak{q}_1 \cap \cdots \cap \mathfrak{q}_n$ with $\mathfrak{q}_i$ primary.
    We say that such a decomposition is \emph{minimal} or \emph{reduced} if:

    \begin{enumerate}
        \item $\mathfrak{p}_i \coloneqq \text{rad}(\mathfrak{q}_i)$ are distinct prime ideals. \label{item:distinct-primes}
        \item $\mathfrak{q}_i \not\subset \bigcap_{j \neq i} \mathfrak{q}_j$ for all $i$. \label{item:noninclusion}
    \end{enumerate}
\end{defn}

\begin{rk}
    Given a primary decomposition of $I$, we can always find a reduced one.
    This is because if the decomposition is not reduced,
    we can decrease the number of primary ideals involved.
    If condition~\ref{item:noninclusion} is not satisfied for some $i$,
    we can remove $\mathfrak{q}_i$ from the decomposition and the equality will still hold.
    If condition~\ref{item:distinct-primes} is not satisfied
    (say, $\mathfrak{p}_i = \mathfrak{p}_j = \mathfrak{p}$),
    then we can replace $\mathfrak{q}_i$ and $\mathfrak{q}_j$ by
    $\mathfrak{q} \coloneqq \mathfrak{q}_i \cap \mathfrak{q}_j$.
    To check that $\mathfrak{q}$ is ($\mathfrak{p}$ -) primary,
    note that (by an extra exercise) $\text{rad}(\mathfrak{q}) = \mathfrak{p}$
    and if $ab \in \mathfrak{q}$ and $a \notin \mathfrak{q}$,
    we can assume $a \notin \mathfrak{q_i} \Rightarrow b \in \mathfrak{p}$.
\end{rk}

\begin{rk}
    However, minimal decompositions are not completely unique.
    In the following propositions, we will produce some invariants of these decompositions.
\end{rk}

\begin{lemma}
    Let $A$ be a ring, let $\mathfrak{q}$ be a $\mathfrak{p}$-primary ideal of $A$ and let $a \in A$.
    If \linebreak ${(\mathfrak{q} : a) \coloneqq \{x \in A \mid ax \in \mathfrak{q}\}}$,
    then:

    \begin{enumerate}
        \item $a \in \mathfrak{q} \Rightarrow (\mathfrak{q} : a) = A$.
        \item $a \notin \mathfrak{q} \Rightarrow (\mathfrak{q} : a)$ is $\mathfrak{p}$-primary.
        \item $a \notin \mathfrak{p} \Rightarrow (\mathfrak{q} : a) = \mathfrak{q}$
    \end{enumerate}

    \begin{proof}
        We will show them in order:
        \begin{enumerate}

            \item
            Trivial.

            \item
            Let us first prove that $\text{rad}\,((\mathfrak{q} : a)) = \mathfrak{p}$.
            We have inclusions $\mathfrak{q} \subseteq (\mathfrak{q} : a) \subseteq \mathfrak{p}$,
            where the second one holds because $ab \in \mathfrak{q}, a \notin q \Rightarrow b \in \mathfrak{p}$ (by definition of $\mathfrak{p}$-primary).
            Taking radicals, we get $\mathfrak{p} \subseteq \text{rad}\,((\mathfrak{q} : a)) \subseteq \mathfrak{p}$.
            Now, let $cd = b \in (\mathfrak{q} : a)$ (that is, $acd \in \mathfrak{q}$),
            and suppose $c \notin (\mathfrak{q} : a)$.
            This means that $ac \notin \mathfrak{q}$, so $d \in \mathfrak{p}$.

            \item
            Suppose that $b \in (\mathfrak{q} : a)$ and $b \notin \mathfrak{q}$.
            Because $ab \in \mathfrak{q}$ and $\mathfrak{q}$ is $\mathfrak{p}$-primary,
            this implies $a \in \mathfrak{p}$.

        \end{enumerate}
    \end{proof}
\end{lemma}


\begin{theorem}
    In the above notation, we have that \[\{\mathfrak{p}_1 \ldots \mathfrak{p}_n\} = \{\text{rad}\,(I : a) \mid a \in A\} \cap \text{Spec}\,A\]

    \begin{proof}

    \end{proof}
\end{theorem}

    \section{Hilbert's Nullstellensatz}\label{sec:hilbert-nullstellensatz}
    \begin{defn}
    Let $K$ be a field and $F = \{f_1, \ldots f_n\} \subset K[T_1, \dots, T_n]$.
    The \emph{algebraic variety} associated to $F$ is
    \[ V(F) = \{ (a_1, \ldots, a_n) \in K^n \, | \, \forall f \in F, f(a_1, \ldots, a_n) = 0 \} \]
    Similarly, given $S \subset K^n$, the \emph{associated ideal} of $S$ is
    \[ I(S) = \{ f \in K[T_1, \ldots, T_n] \, | \, \forall (a_1, \ldots, a_n) \in S, f(a_1, \ldots, a_n) = 0 \} \]
\end{defn}

\begin{rk}
    $V(\{f_1, \ldots f_n\}) = V((f_1, \ldots f_n))$.
    This gives us a way to go from ideals to subsets of $K^n$ and vice versa.
    As we will see in the following examples, this correspondence is not bijective.
    However, it is clear that $S \subseteq V(I(S))$ and $F \subseteq I(V(F))$.
    Also, $S_1 \subseteq S_2 \Rightarrow V(S_1) \supseteq V(S_2)$ and
    $I_1 \subseteq I_2 \Rightarrow V(I_1) \supseteq V(I_2)$.
\end{rk}

\begin{prop}
    V(I(V(F))) = V(F) and I(V(I(S))) = I(S).
    \begin{proof}
        The two proofs are analogous, so we will only prove the first one.
        We have seen I(V(F)) $\supseteq$ F so V(I(V(F))) $\subseteq$ V(F).
        However, if we set $S = V(F)$, $VI(S) \supseteq S$, that is, $V(I(V(F))) \supseteq V(F)$.
    \end{proof}
\end{prop}

\begin{example}
    $V$ and $I$ are not mutually inverse in general.
    For example, let us set $n=1$ and $K = \mathbb{R}$.
    \begin{itemize}
        \item $IV((x^2)) = I(\{0\}) = (x)$.
        \item $IV((x^2+1)) = I(\emptyset) = \mathbb{R}[x]$.
        \item $VI(\mathbb{R} \setminus \{0\}) = V((0)) = \mathbb{R}$.
    \end{itemize}
\end{example}

\begin{rk}
    The third issue is solved by just restricting the domain of $V$ to the algebraic subsets of $K^n$.
    $VI(s)$ is the smallest algebraic subset of $K^n$ containing $S$ and is called the \emph{Zariski closure} of $S$.
    The first and second issues would be solved by restricting ourselves to \emph{radical} ideals
    and \emph{algebraically closed} fields, respectively.
    The remainder of this section will be spent on showing that.
\end{rk}

\begin{prop}
    $V(S)$ is always a radical ideal.
    \begin{proof}
        Suppose that $f^n \in I(S).$
        Then,
        $\forall (a_1, \ldots, a_n) \in S, 0 = f^n(a_1, \ldots, a_n)
        = f(a_1, \ldots, a_n)^n \Rightarrow f(a_1, \ldots, a_n) = 0$.
        Thus, $f \in I(S)$ and $I(S)$ is radical.
    \end{proof}
\end{prop}

For a moment, we would like to study the maximal ideals of $K$ in relation to the algebraic subsets of $K^n$.

\begin{prop}
    Let $p$ = ($p_1, \ldots, p_n$) be a point in $K^n$.
    Then, the ideal $\mathfrak{m}_p \coloneqq I(\{p\})
    = (T_1 - p_1, \ldots, T_n - p_n) \subset K[T_1, \ldots T_n]$
    is maximal.

    \begin{proof}
        The maximality of $(T_1 - p_1, \ldots, T_n - p_n)$
        comes from the fact that $K[T_1, \ldots T_n]/(T_1 - p_1, \ldots, T_n - p_n) \cong K[p_1, \ldots p_n] = K$
        is a field.
        Now, clearly
        $I(\{p\}) \supseteq (T_1 - p_1, \ldots, T_n - p_n)$
        and
        $1 \notin I(\{p\})$.
        By the maximality of the ideal we found, we must have equality.

    \end{proof}

\end{prop}

\begin{rk}
    However, the converse is not true.
    That is, we can find maximal ideals that are not of the form $\mathfrak{m}_p$.
\end{rk}

\begin{example}
    Let $f$ be any irreducible polynomial in $K[T]$
    of degree $> 1$ (for example, $f = T^2 + 1$ in $\mathbb{R}[T]$).
    Then, $(f)$ is a maximal ideal of $K[T]$ because
    polynomial rings over fields are P.I.D.s.
    Obviously this can't happen in algebraically closed fields.
    We will prove that in fact in algebraically closed fields
    all maximal ideals correspond to points in $K^n$.
\end{example}

\begin{theorem}[Weak Nullstellensatz]
    Any finitely generated $K$-algebra $A$ that is a field is a finite extension of $K$.
    \begin{proof}
        TODO
    \end{proof}
\end{theorem}

\begin{corollary}
    If $K$ is algebraically closed,
    Any finitely generated $K$-algebra $A$ that is a field is $K$.

    \begin{proof} it must be
    generated by algebraic elements over $K$,
    which lie in $K$.
    therefore $A = K$.
    \end{proof}
\end{corollary}

\begin{corollary}
    If $K$ is algebraically closed,
    the only maximal ideals of $K[T_1, \ldots T_n]$ are of the form $\mathfrak{m}_p$.
    \begin{proof}
        Let $\mathfrak{m}$ be a maximal ideal of $K[T_1, \ldots T_n]$.
        Then, $K[T_1, \ldots T_n]/\mathfrak{m}$ is a field so it is just $K$.
        Let $p_i \in K$ be the projection of $T_i$ in $K[T_1, \ldots T_n]/\mathfrak{m}$.
        Then, $\mathfrak{m} \supseteq (T_1 - p_1, \ldots, T_n - p_n) = \mathfrak{m}_p$.
        By maximality of both, we must have equality.
    \end{proof}
\end{corollary}

\begin{theorem}[Hilbert's Nullstellensatz]
    Let $K$ be an algebraically closed field.
    Let $J$ be an ideal of $K[T_1, \ldots T_n]$.
    Then, $I(V(J)) = \text{rad}\,(J)$.

    \begin{proof}
        We will prove this by double inclusion.

        \begin{itemize}
            \item [$\subseteq$] is clear because $IV(J)$ is a radical ideal that contains $J$.
            \item[$\supseteq$]: Let $J = (g_1, \ldots g_m)$ (it is finitely generated by Hilbert's basis theorem). \linebreak
            Let $h \in I(V(J))$.
            We may assume $h \neq 0$.
            Consider the ideal $J' \coloneqq (g_1, \ldots g_m, 1 - hS) \subset K[T_1, \ldots T_{n}, S]$.
            Because $h$ vanishes on $V(J)$, so does $hS$ on $V(J)_{K^{n+1}}$.
            This means that $V(J') = \emptyset$.
            This means that $J' = K[T_1, \ldots T_{n}, S]$.
            Otherwise, $J' \subset \mathfrak{m}_p$ for some maximal ideal $\mathfrak{m}_p$,
            so it would vanish at $p \in K^{n+1}$.
            Therefore, we can write:
            \[
                1 = \sum_{i=1}^m f_i g_i + (1 - hS)f
            \]
            Including in the total ring of fractions, and making the substitution $S = 1/h$,
            and taking common denominators, we get:
            \[
                1 = \frac{1}{h^N} \sum_{i=1}^m \tilde{f_i} g_i \Rightarrow h^N = \sum_{i=1}^m \tilde{f_i} g_i
            \]
            for some $N \in \mathbb{N}$ and $\tilde{f_i} \in K[T_1, \ldots T_n]$.
            That is, $h \in \text{rad}\,(J)$.

        \end{itemize}
    \end{proof}

\end{theorem}




\end{document}