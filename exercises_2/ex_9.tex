\begin{problem}\textcolor{white}{.}

    \begin{enumerate}[label=(\theproblem.\arabic*),ref=\theproblem.\arabic*]
        \item Let be $A$ be a local ring, $M$, $N$ finitely generated $A$-modules.
        Prove that $M \otimes_A N = 0$ if and only if $M = 0$ or $N = 0$.
            \begin{sol}
                \begin{proof}
                    For the left implication, suppose without loss of generality that $M = 0$.
                    For any $n \in N$, $0 \otimes_A n = (0\cdot0) \otimes_A n = 0(0 \otimes_A n) = 0$.

                    For the right implication, consider the maximal ideal $\mathfrak{m}$ of $A$
                    and the quotient $\mathbb{K}\coloneqq~A/\mathfrak{m}$.
                    \begin{claim}
                        For any finitely generated $A$-module $P$, 
                        $P \otimes_A \mathbb{K}$ is a $\mathbb{K}$-vector space.
                        \begin{proof}
                            We only need to check that the scalar multiplication is well-defined.
                            Indeed, if $q-r \in \mathfrak{m}$, and $p\,\otimes_A\,k$ one of the generators of $P \otimes_A \mathbb{K}$,
                            then $q\,(p \otimes_A k) - r\,(p \otimes_A k) =
                            (p \otimes_A qk) - (p \otimes_A rk) =
                            p \otimes_A (qk - rk) =
                            p \otimes_A (q-r)k =
                            p \otimes_A 0 = 0$.
                            Therefore, $ q\,(p \otimes_A k) = r\,(p \otimes_A k)$.
                            Because this is true for any generator, it is also true for any finite linear combination of generators (that is, the whole tensor product).
                        \end{proof}
                    \end{claim}

                    \begin{rk}
                        In fact, we can express all elements of $P \otimes_A \mathbb{K}$ as $p \otimes_A 1$
                        for some $p \in P$ and $k \in \mathbb{K}$.
                        This is true because we have a set of generators
                        of the form $p \otimes_A k$ with $p \in P$ and $k \in \mathbb{K}$, and
                        $p_1 \otimes_A k_1 + p_2 \otimes_A k_2 =
                        k_1 (p_1 \otimes_A 1 ) + k_2 (p_2 \otimes_A 1) =
                        (\widehat{k_1} p_1 + \widehat{k_2} p_2) \otimes_A 1$,
                        where $\widehat{k_i}$ is any lift of $k_i$ to $A$.
                    \end{rk}

                    We have just shown that $M \otimes_A \mathbb{K}$ and $N \otimes_A \mathbb{K}$ have a $\mathbb{K}$-vector space structure.
                    It is clear that $M \otimes_A N = 0$ implies $M \otimes_A  N \otimes_A \mathbb{K} = 0$.
                    In fact, we will show that it is a certain $\mathbb{K}$-tensor product, from which the result will follow:

                    \begin{claim}
                        $M \otimes_A N \otimes_A \mathbb{K} \cong (M \otimes_A \mathbb{K}) \otimes_{\mathbb{K}} (N \otimes_A \mathbb{K})$ as $\mathbb{K}$-vector spaces.
                            \begin{proof}

                                We will show the universal property of the tensor product for the map
                                \begin{align*}
                                    \varphi: (M \otimes_A \mathbb{K}) \times (N \otimes_A \mathbb{K}) \rightarrow M \otimes_A N \otimes_A \mathbb{K} \\
                                    (m \otimes_A 1, n \otimes_A 1) \mapsto m \otimes_A n \otimes_A 1 \\
                                \end{align*}

                                \begin{itemize}
                                    \item \emph{$\varphi$ is well-defined}: Suppose $m \otimes_A 1 = m' \otimes_A 1$.
                                    then, $(m - m') \otimes_A 1 = 0 = 0 \otimes_A 1$




                                    \item \emph{$\varphi$ is $\mathbb{K}$-bilinear} because I say so (si lo miras lo ves).
                                    \item Let $V$ be a $\mathbb{K}$-vector space and $f: (M \otimes_A \mathbb{K}) \times (N \otimes_A \mathbb{K}) \rightarrow V$ a $\mathbb{K}$-bilinear map.
                                        We will show that there exists a unique $\mathbb{K}$-linear map $\overline{f}: M \otimes_A N \otimes_A \mathbb{K} \rightarrow V$ such that $f = \overline{f} \circ \varphi$.
                                        Clearly, if such map exists, it is unique by construction (the image of $m \otimes_A n \otimes_A 1$ is just $f(m \otimes_A 1, n \otimes_A 1)$).
                                        Let us show that this map $\bar{f}$ is well-defined and $\mathbb{K}$-linear.
                                        \begin{itemize}
                                            \item \emph{$\overline{f}$ is well-defined}: Suppose $m \otimes_A n \otimes_A 1 = m' \otimes_A n' \otimes_A 1$.
                                            \item \emph{$\overline{f}$ is $\mathbb{K}$-linear} because I say so (si lo miras lo ves).
                                        \end{itemize}

                                \end{itemize}

                            \end{proof}

                    \end{claim}

                \end{proof}
            \end{sol}
        \item Prove that the result is no longer true if $A$ is not local.
            \begin{sol}
                \begin{claim}
                    Let $\mathfrak{m}_1$ and $\mathfrak{m}_2$ be two distinct maximal ideals of $A$.
                    Then, $A/\mathfrak{m}_1 \otimes_A A/\mathfrak{m}_2 = 0$.
                    \begin{proof}
                        We will show that all generators $\hat{a} \otimes_A \bar{b}$ of the tensor product are zero,
                        where $\hat{a}$ and $\bar{b}$ are the images of $a$ and $b$ in the quotients.
                        Let $x \in \mathfrak{m}_1 \setminus \mathfrak{m}_2$ and let $\bar{y} = (\bar{x})^{-1}$.
                        Then $\hat{a}\otimes_A\bar{b}=\hat{a}\otimes_A\overline{bxy} = \widehat{xa}\otimes_A\overline{yb} = \hat{0}\otimes_A\overline{yb}=0$.


                    \end{proof}
                \end{claim}
            \end{sol}
    \end{enumerate}
\end{problem}