\begin{problem}\textcolor{white}{.}

    \begin{enumerate}[label=(\theproblem.\arabic*),ref=\theproblem.\arabic*]
        \item Let be $A$ be a local ring, $M$, $N$ finitely generated $A$-modules.
        Prove that $M \otimes_A N = 0$ if and only if $M = 0$ or $N = 0$.
            \begin{sol}
                \begin{proof}
                    For the left implication, suppose without loss of generality that $M = 0$.
                    For any $n \in N$, $0 \otimes_A n = (0\cdot0) \otimes_A n = 0(0 \otimes_A n) = 0$.

                    For the right implication, consider the maximal ideal $\mathfrak{m}$ of $A$
                    and the quotient $\mathbb{K}\coloneqq~A/\mathfrak{m}$.
                    \begin{claim}
                        For any finitely generated $A$-module $P$, 
                        $P \otimes_A \mathbb{K} \cong P/\mathfrak{m}P$ is a $\mathbb{K}$-vector space.
                        \begin{proof}
                            The first isomorphism is a consequence of Extra Exercise 14.
                            We only need to check that the scalar multiplication is well-defined.
                            Indeed, if $a - b \in \mathfrak{m}, p \in P, ap - bp = (a-b) p \in \mathfrak{m}P$
                            so $ap + \mathfrak{m}P = bp + \mathfrak{m}P$.
                        \end{proof}
                    \end{claim}

                    We have just shown that $M \otimes_A \mathbb{K}$ and $N \otimes_A \mathbb{K}$ have a $\mathbb{K}$-vector space structure.
                    It is clear that $M \otimes_A N = 0$ implies $(M \otimes_A N) / \mathfrak{m}(M \otimes_A N) = M \otimes_A  N \otimes_A \mathbb{K} = 0$.
                    In fact, we will show that it is a certain $\mathbb{K}$-tensor product, from which the result will follow:

                    \begin{claim}
                        $M \otimes_A N \otimes_A \mathbb{K} \cong (M \otimes_A \mathbb{K}) \otimes_{\mathbb{K}} (N \otimes_A \mathbb{K})$ as $\mathbb{K}$-vector spaces.
                            \begin{proof}

                                Let us simplify the notation by writing $M/\mathfrak{m}M \eqqcolon M/\mathfrak{m}$, and the same for the other modules.

                                We will show that $M/\mathfrak{m} \otimes_\mathbb{K} N/\mathfrak{m} \cong (M \otimes_A N)/\mathfrak{m}$ via the universal property of the quotient.
                                For this, we will first construct a candidate linear projection \linebreak
                                ${\hat{\pi}: M \otimes_A N \rightarrow M/\mathfrak{m} \otimes_\mathbb{K} N/\mathfrak{m}}$.
                                We have the usual tensor product bilinear maps
                                \[\varphi: M \times N \rightarrow M \otimes_A N\]
                                and
                                \[\overline{\varphi}: M/\mathfrak{m} \times N/\mathfrak{m} \rightarrow M/\mathfrak{m} \otimes_\mathbb{K} N/\mathfrak{m}\]
                                Furthermore, consider the canonical projections
                                \begin{align*}
                                    \pi_M: M &\rightarrow M/\mathfrak{m}\\
                                    \pi_N: N &\rightarrow N/\mathfrak{m}
                                \end{align*}
                                Clearly,
                                \[\overline{\varphi} \circ (\pi_M \times \pi_N): M \times N \rightarrow M/\mathfrak{m} \otimes_{\mathbb{K}} N/\mathfrak{m}\]
                                is $A$-bilinear, so, by the universal property of the tensor product, there exists a unique $A$-linear map $\hat{\pi}$ making the following diagram commute:

                                \begin{center}
                                    \begin{tikzcd}
                                        {M \times N} && {M \otimes_A N} \\
                                        \\
                                        {M / \mathfrak{m} \times N / \mathfrak{m}} && {M / \mathfrak{m} \otimes_{\mathbb{K}} N / \mathfrak{m}}
                                        \arrow["\varphi", from=1-1, to=1-3]
                                        \arrow["{\hat{\pi}}", from=1-3, to=3-3]
                                        \arrow["{\pi_M\times\pi_N}"', from=1-1, to=3-1]
                                        \arrow["{\overline{\varphi}}"', from=3-1, to=3-3]
                                    \end{tikzcd}
                                \end{center}

                                Now, consider any $A$-linear map $f: M \otimes_A N \rightarrow P$ such that $f(\mathfrak{m}(M \otimes_A N)) = 0$.
                                We want to see that it factors $A$-linearly and uniquely through $\hat{\pi}$ as:

                                \begin{center}
                                    \[\begin{tikzcd}
                                    {M \times N} && {M \otimes_A N} && P \\
                                    \\
                                    {M / \mathfrak{m} \times N / \mathfrak{m}} && {M / \mathfrak{m} \otimes_{\mathbb{K}} N / \mathfrak{m}}
                                    \arrow["\varphi", from=1-1, to=1-3]
                                    \arrow["{\hat{\pi}}", from=1-3, to=3-3]
                                    \arrow["{\pi_M\times\pi_N}"', from=1-1, to=3-1]
                                    \arrow["{\overline{\varphi}}"', from=3-1, to=3-3]
                                    \arrow["f", from=1-3, to=1-5]
                                    \arrow["{\bar{f}}", from=3-3, to=1-5]
                                \end{tikzcd}\]

                                \end{center}

                                The only way to construct
                                $\bar{f}$ is by setting \[{\bar{f}(\overline{m_i} \otimes_{\mathbb{K}} \overline{n_j}) = f(m_i \otimes_A n_j)}\]
                                on a $K$-basis $\{\overline{m_i} \otimes_{\mathbb{K}} \overline{n_j}\}$ and extending $A$-linearly (considering $\mathfrak{m} \otimes_{\mathbb{K}} N / \mathfrak{m}$ as an $A$-module):

                                \[
                                    \bar{f} \left( \sum_{i,j} \overline{\lambda^i_j} \overline{m_i} \otimes_\mathbb{K} \overline{n_j} \right) =
                                    \bar{f} \left( \sum_{i,j} \lambda^i_j \overline{m_i} \otimes_\mathbb{K} \overline{n_j} \right) =
                                    \sum_{i,j} \lambda^i_j f(m_i \otimes_A n_j)
                                \]

                                This well-defined because:
                                \begin{itemize}
                                    \item If $\overline{m_i'} = \overline{m_i}$ and $\overline{n_j'} = \overline{n_j}$, then
                                    \[f(m_i' \otimes_A n_j') - f(m_i \otimes_A n_j) = f((m_i' - m_i) \otimes_A n_j') + f(m_i \otimes_A (n_j' - n_j)) = 0\]
                                    so any choices of $m_i \otimes_A n_j$ on the right give the same result.
                                    \item if ${\lambda \in \mathfrak{m}}$, ${x \in M \otimes_A N}$,
                                    \[\lambda f(x) = f(\lambda x) = 0\]
                                    so that any choices of $\lambda^i_j$s on the right yields the same result.
                                \end{itemize}




                                Also, it is $A$-linear by construction.
                                Verifying that the diagram commutes is a matter of applying linearity of $f$, $\bar{f}$
                                $\varphi$ and $\overline{\varphi}$, and the commutativity of the diagram for $\hat{\pi}$.
                                $\varphi$ and $\overline{\varphi}$, and the commutativity of the diagram for $\hat{\pi}$.

                                Now, we have shown that $M/\mathfrak{m} \otimes_\mathbb{K} N/\mathfrak{m} \cong (M \otimes_A N)/\mathfrak{m}$ as $A$-modules.
                                But both are $\mathbb{K}$-vector spaces, so they are isomorphic as $\mathbb{K}$-vector spaces (for any $A$-linear function $g$,
                                $\overline{\lambda} = \overline{\lambda'}$,
                                $g(\lambda x) - g(\lambda'x) = (\lambda - \lambda') g(x) = 0x = 0$ so $g$ is also $\mathbb{K}$-linear).
                            \end{proof}
                    \end{claim}

                    So far, we have that $ (M \otimes_A \mathbb{K}) \otimes_\mathbb{K}( N \otimes_A \mathbb{K}) = 0$.
                    However, from linear algebra we know that the dimension of a tensor product over a vector space
                    is the product of the dimensions of the factors.
                    Therefore, $M \otimes_A \mathbb{K} = 0$ or $N \otimes_A \mathbb{K} = 0$.
                    Now, recall that $M \otimes_A \mathbb{K} = M \otimes_A A/\mathfrak{m} $, and the same for $N$.
                    $\mathfrak{m}$ is the only maximal ideal of $A$ so in particular it is contained in the Jacobson radical (it fact, it \emph{is} the Jacobson radical).
                    By Problem~\ref{itm:2_1}, $M = 0$ or $N = 0$, as we wanted to show.



                \end{proof}
            \end{sol}
        \item Prove that the result is no longer true if $A$ is not local.
            \begin{sol}
                \begin{claim}
                    Let $\mathfrak{m}_1$ and $\mathfrak{m}_2$ be two distinct maximal ideals of $A$.
                    Then, $A/\mathfrak{m}_1 \otimes_A A/\mathfrak{m}_2 = 0$.
                    \begin{proof}
                        We will show that all generators $\hat{a} \otimes_A \bar{b}$ of the tensor product are zero,
                        where $\hat{a}$ and $\bar{b}$ are the images of $a$ and $b$ in the quotients.
                        Let $x \in \mathfrak{m}_1 \setminus \mathfrak{m}_2$ and let $\bar{y} = (\bar{x})^{-1}$.
                        Then $\hat{a}\otimes_A\bar{b}=\hat{a}\otimes_A\overline{bxy} = \widehat{xa}\otimes_A\overline{yb} = \hat{0}\otimes_A\overline{yb}=0$.


                    \end{proof}
                \end{claim}
            \end{sol}
    \end{enumerate}
\end{problem}