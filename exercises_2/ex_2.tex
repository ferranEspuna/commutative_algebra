\begin{problem}
    Under the previous hypothesis, prove:
    \begin{enumerate}[label=(\theproblem.\arabic*),ref=\theproblem.\arabic*]
        \item \label{itm:2_1} $A/I \otimes_A M = 0 \Rightarrow M = 0$.
            \begin{sol}
                \begin{proof}
                    The solution comes directly from the Extra Exercise $14$ which states:
                    \begin{claim}
                        Let $A$ be a ring, $M$ an $A$-module and $I \subseteq A$ an ideal.
                        Then
                        \[
                            M/IM \cong A/I \otimes_A M
                        \]
                    \end{claim}
                    Follows the claim that
                    \[
                        M/IM \cong A/I \otimes_A M = 0 \Rightarrow M/IM = 0 \Rightarrow M = IM
                    \]
                    Finally, using Problem~\ref{itm:nakayamas_lemma}
                    \[
                        M = IM \Rightarrow M = 0
                    \]
                \end{proof}
            \end{sol}
        \item \label{itm:2_2} If $N \subseteq M$ is a submodule, $M = IM + N \Rightarrow M = N$.
            \begin{sol}
                \begin{proof}
                    Suppose $M = IM + N$.
                    Then
                    \[
                        M / N = (IM + N) / N \cong I (M / N)
                    \]
                    where the isomorphism comes from projecting $IM+N$ to $M/N$
                    and checking that the image is $I (M/N)$.
                    We can use Problem~\ref{itm:nakayamas_lemma}
                    \[
                        M/N = I (M / N) \Rightarrow M/N = 0
                    \]
                    Thus, $M = N$ and we have proven the statement.
                \end{proof}
            \end{sol}
        \item If $f: N \rightarrow M$ is a homomorphism, $\overline{f}: N/IN \rightarrow M/IM$ surjective $\Rightarrow$ $f$ surjective.
            \begin{sol}
                \begin{proof}
                    \begin{claim}
                        $M = IM + f(N)$
                        \begin{proof}
                            We will prove both inclusions:
                            \begin{itemize}
                                \item[$(\supseteq)$] Comes directly from the fact that $IM \subseteq M$ and $f(N) \subseteq M$ by definition.
                                \item[$(\subseteq)$] Take $m \in M$.
                                Since $\overline{f}$ is surjective, exists $\overline{n} \in N/IN$ such that $\overline{f}(\overline{n}) = \overline{m}$.
                                By definition, $\overline{f(n)} = \overline{m}$ and thus $f(n) = m + x$ for some $x \in IM$.
                                Since $f(n) \in f(N)$ and $x \in IM$ we conclude that
                                \[
                                    m = f(n) + x \in IM + f(N)
                                \]
                            \end{itemize}
                        \end{proof}
                    \end{claim}
                    Since $f(N) \subseteq M$ and $M = IM + f(N)$ we can use~\ref{itm:2_2} to conclude that $M = f(N)$ and thus, $f$ is surjective.
                \end{proof}
            \end{sol}
    \end{enumerate}
\end{problem}