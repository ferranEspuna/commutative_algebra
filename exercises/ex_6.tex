







% PROBLEM 6
\begin{problem}\textcolor{white}{this is so that the items appear below the problem title}
    \begin{enumerate}[label=(\theproblem.\arabic*),ref=\theproblem.\arabic*]
        \item Let $f(T) \in K[T]$ be irreducible where $K$ is a field.
        Prove that the ideal $(f(T))$ is maximal.
            \begin{sol}
                \begin{proof}
                    Consider $I = (f(T))$ and $J$ such that $I \subseteq J \neq K[T]$.
                    \begin{claim} \label{itm:polynomial_field_is_pid}
                        If $K$ is a field, then $K[T]$ is a P.I.D.
                        \begin{proof}
                            Take $I$ ideal in $K[T]$ and the polynomial of minimal degree $p(T) \neq 0$ in $I$.
                            Then, $d(p(T)) > 0$, otherwise $p(T) \in K$ and $p p^{-1} = 1 \in I$ (in contradiction with $I \neq K[T]$).
                            Now, consider $g(T) \in I$.
                            By the division algorithm, $\exists q(T), r(T) \in K[T]$ such that $g(T) = p(T) q(T) + r(T)$ with $d(r(T)) < d(p(T))$.
                            Then $r(T) = g(T) - p(T) q(T) \in I$ since $g(T), p(T) \in I$.
                            By minimality of $p(T) \neq 0$, we have that $r(T) = 0$.
                            Thus, $g(T) = p(T) q(T)$, and $I = (p(T))$.
                        \end{proof}
                    \end{claim}
                    For the claim we know that $K[T]$ is a P.I.D., thus $\exists h(T) \in K(T)$ such that $J = (h(T))$.
                    Then
                    \[ f(T) \in (f(T)) \subseteq J = (h(T)) \]
                    This implies that $f(T) = h(T) g(T)$, but since $f(T)$ is irreducible, either $h(T)$ or $g(T)$ is an element of $K$.
                    $h(T) \notin K$, otherwise $J = K[T]$ which is not a field.
                    Then $g(T) \in K$ and $h(T) = f(T) a^{-1}$.
                    We conclude that $J \subset I$, so $I$ is maximal.
                \end{proof}
            \end{sol}
        \item Describe the spectrum of $\mathbb{R}[T], \mathbb{C}[T], \mathbb{R}[T]/(T^2+9), \mathbb{C}[T]/(T^2+9)$
            \begin{sol}
                    By the first part of the exercise, we know $\mathbb{R}[T]$ and $\mathbb{C}[T]$ are P.I.D.s,
                    and in particular all their nonzero prime ideals are maximal.
                    Thus, the spec of $\mathbb{R}[T]$ and $\mathbb{C}[T]$ consists
                    of the principal ideals generated by irreducible polynomials.
                    The irreducible polynomials in $\mathbb{R}[T]$ are well known, so:
                    \begin{itemize}
                        \item $\text{Spec}\,\mathbb{R}[T] = \{ (0)\} \cup \{ (T-c) \, | \, c \in \mathbb{R} \}
                        \cup \{(T^2 + bT + c) \, | \, b, c \in \mathbb{R}, \, b^2-4c < 0\}$
                        \item $\text{Spec}\,\mathbb{C}[T] = \{ (0)\} \cup \{ (T-c) \, | \, c \in \mathbb{C} \}$
                    \end{itemize}
                    This observation also helps us describe the spectrum of the quotients.
                    We know, for any ring $R$ and ideal $I$,
                    there is a bijection between the ideals of $R/I$ and the ideals of $R$ that contain $I$,
                    given by the projection $\pi: R \rightarrow R/I$.
                    Furthermore, all prime ideals in $R/I$ are of the form $\pi(\mathfrak{p})$ for some prime ideal $\mathfrak{p}$
                    of $R$ that containing $I$. $T^2 + 9$ is irreducible in $\mathbb{R}[T]$,
                    but it factorizes (uniquely, as P.I.D $\Rightarrow$ U.F.D) as $(T+3i)(T-3i)$ in $\mathbb{C}[T]$.
                    In the first case, the ideal is maximal, so
                    \[
                        \text{Spec}\,\mathbb{R}[T]/(T^2+9) = \{ (0)\}
                    \]
                    Equivalently, $\mathbb{R}[T]/(T^2+9)$ is a field.
                    In the second case, the only prime ideals in $\mathbb{C}$ properly containing
                    $(T^2+9)$ are $(T + 3i)$ and $(T - 3i)$, so
                    \[
                        \text{Spec}\,\mathbb{C}[T]/(T^2+9) = \{ (0), (T + 3i), (T - 3i) \}
                    \]

            \end{sol}
    \end{enumerate}
\end{problem}