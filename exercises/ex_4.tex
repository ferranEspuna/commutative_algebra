% PROBLEM 4
\begin{problem}
    Let $A$ be a ring such that every ideal not contained in the nilradical $\mathcal{N}(A)$ has a non-zero idempotent element (i.e. an element $e \neq 0$ such that $e^2 = e$).
    Prove that the nilradical and the Jacobson radical $\mathcal{J}(A)$ are equal.
    \begin{sol}
        \begin{proof}
            We will prove the two inclusions separately:
                \begin{itemize}
                    \item[$(\supseteq)$]
                    (Same as in~\ref{itm:nilradical_equals_jacobson_in_polynomial_ring})
                    \[ \mathcal{N}(A) = \bigcap_{I \in \text{Spec}\,A} I \subseteq \bigcap_{I \in \text{Max}\,A} I = \mathcal{J}(A) \]
                    The first equality follows~\ref{itm:nilradical_is_intersection_of_all_primes}, while the inclusion is a consequence of $\text{Max}\,A \subseteq \text{Spec}\, A $.
                    \item[$(\subseteq)$]
                    Suppose that $\mathcal{J}(A) \nsubseteq \mathcal{N}(A)$.
                    Then, since $\mathcal{J}(A)$ is an intersection of ideals and thus an ideal itself, we have that $\exists e \in \mathcal{J}(A)$ such that $e^2 = e \neq 0$.
                    By~\ref{itm:jacobson_element_to_invertible}, we have that $1 - e 1$ is invertible, so
                    \[ (1 + e) = (1 - e)^{-1}(1 - e)(1 + e) = (1 - e)^{-1}(1 - e^2) = (1 - e)^{-1}(1 - e) = 1 \]
                    Finally, we get that $e = 0$, in contradiction with the definition of $e$.
                    Thus, we conclude that $\mathcal{J}(A) \subseteq \mathcal{N}(A)$.
                \end{itemize}
        \end{proof}
    \end{sol}
\end{problem}