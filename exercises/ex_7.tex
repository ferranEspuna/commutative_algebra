\begin{problem}
Describe $\text{spec}(\mathbb{Z}[T])$
\begin{sol}
    Suppose that $\mathfrak{p} \subset \mathbb{Z}[T]$ is an ideal.
    We will distinguish three cases:
    \begin{enumerate}
        \item $\mathfrak{p} = (0)$.
        \textcolor{red}{$(0)$} is a prime ideal of $\mathbb{Z}[T]$ since $\mathbb{Z}[T]$ is an integral domain.
        \item $\mathfrak{p} = (f)$ is a principal ideal generated by a non-zero polynomial $f \in \mathbb{Z}[T]$.
        There are two possibilities:
        \begin{itemize}

            \item $\deg f = 0 \Leftrightarrow f = n \in \mathbb{Z}$.
            Suppose $(n)$ is a prime ideal.
            In this case, it is clear by the inclusion $\varphi:\mathbb{Z} \to \mathbb{Z}[T]$ that $(n) = (n)^c$ is prime in $\mathbb{Z}$.
            Conversely, let $p \in \mathbb{Z}$ be a prime number. $\mathbb{Z}[T]/(p) \cong \mathbb{Z}/(p)[T]$
            is an integral domain since $\mathbb{Z}/(p)$ is a field (in particular, an integral domain).
            We obtain the ideals \textcolor{red}{$(p)$ where $p \in \mathbb{Z}$ is prime}.

            \item $\deg f > 0$.
            \begin{claim}
                $(f)$ is prime $\Rightarrow f$ is primitive.
                \begin{proof}
                    Otherwise, there is some non-unit $p \in \mathbb{Z}$ that divides all coefficients of $f$.
                    Then $f = pg$ for some $g \in \mathbb{Z}[T]$, which is a non-trivial factorization of $f$.
                    $f$ is not irreducible, so it is not prime (since $\mathbb{Z}$ is a domain).
                \end{proof}
            \end{claim}
            \begin{claim}
                $(f)$ is prime $\Rightarrow f$ is irreducible in $\mathbb{Q}[T]$.
                \begin{proof}
                    Suppose $f = gh$ is a non-trivial factorization, for some $g,h \in \mathbb{Q}[T]$ (in particular, since all non-zero elements of $\mathbb{Q}$ are units, $\deg g, \deg h > 0 \Rightarrow \deg g, \deg h < \deg f$).
                    Taking a common denominator $n$ and multiplying by it, we obtain $nf = \tilde{g}\tilde{h}$ in $\mathbb{Z}[T]$.
                    By the left side, this belongs to $(f)$, but $\tilde{g}, \tilde{h}$ do not belong to $(f)$ since they have strictly smaller degree than $f$.
                \end{proof}
            \end{claim}

            \begin{claim}
                These conditions are also sufficient: if $f$ is irreducible in $\mathbb{Q}[T]$ and primitive, then $(f)$ is prime.
                \begin{proof}
                    Consider the natural inclusion $\phi:\mathbb{Z}[T] \to \mathbb{Q}[T]$.
                    We will extend and then contract the ideal $(f)$ of $\mathbb{Z}[T]$.
                    By taking a common denominator and factors:
                    \[
                        (f)^{ec} = \mathbb{Z} \cap (f)_{\mathbb{Q}} = \mathbb{Z}[T] \cap \left\{\frac{n}{m}fh \mid n, m \in \mathbb{Z}, h \in \mathbb{Z}[T] \, \text{primitive}\right\}
                    \]
                    We know that $(f) \subset (f)^{ec}$.
                    We will prove the other inclusion and therefore equality.
                    Indeed, take $g = \frac{n}{m}fh \in (f)^{ec}$ of the form described above.
                    For any prime $p \mid m$, we have shown that $(p)$ is prime in $\mathbb{Z}[T]$.
                    Furthermore, $f, h$ are primitive, so $p$ does not divide them $\Rightarrow p \mid n$.
                    We can cancel $p$ in the fraction, and repeat the process for all prime factors of $m$. 
                    Therefore, $g \in (f)_{\mathbb{Z}}$.
                    Now, as $f$ is irreducible in $\mathbb{Q}[T]$, which is a PID, $(f)_{\mathbb{Q}} = (f)_{\mathbb{Z}}^e$ is prime in $\mathbb{Q}[T]$.
                    Therefore, so is $(f) = (f)^{ec} \subset \mathbb{Z}[T]$.
                \end{proof}
            \end{claim}
            With all these claims, this case yields ideals of the form \textcolor{red}{$(f)$ where $f \in \mathbb{Z}[T]$ is irreducible and primitive}.

        \end{itemize}
        \item $\mathfrak{p}$ is not principal.
        I don't know how to do this case.
        Idea: Consider the gcd of all the generators in $\mathbb{Q}[T]$,
        we have bezout coefficients, so we can write it as a linear combination of them with rational coefficients.

    \end{enumerate}
\end{sol}
\end{problem}