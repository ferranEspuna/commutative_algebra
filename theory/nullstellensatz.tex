\begin{defn}
    Let $K$ be a field and $F = \{f_1, \ldots f_n\} \subset K[T_1, \dots, T_n]$.
    The \emph{algebraic variety} associated to $F$ is
    \[ V(F) = \{ (a_1, \ldots, a_n) \in K^n \, | \, \forall f \in F, f(a_1, \ldots, a_n) = 0 \} \]
    Similarly, given $S \subset K^n$, the \emph{associated ideal} of $S$ is
    \[ I(S) = \{ f \in K[T_1, \ldots, T_n] \, | \, \forall (a_1, \ldots, a_n) \in S, f(a_1, \ldots, a_n) = 0 \} \]
\end{defn}

\begin{rk}
    $V(\{f_1, \ldots f_n\}) = V((f_1, \ldots f_n))$.
    This gives us a way to go from ideals to subsets of $K^n$ and vice versa.
    As we will see in the following examples, this correspondence is not bijective.
    However, it is clear that $S \subseteq V(I(S))$ and $F \subseteq I(V(F))$.
    Also, $S_1 \subseteq S_2 \Rightarrow V(S_1) \supseteq V(S_2)$ and
    $I_1 \subseteq I_2 \Rightarrow V(I_1) \supseteq V(I_2)$.
\end{rk}

\begin{prop}
    V(I(V(F))) = V(F) and I(V(I(S))) = I(S).
    \begin{proof}
        The two proofs are analogous, so we will only prove the first one.
        We have seen I(V(F)) $\supseteq$ F so V(I(V(F))) $\subseteq$ V(F).
        However, if we set $S = V(F)$, $VI(S) \supseteq S$, that is, $V(I(V(F))) \supseteq V(F)$.
    \end{proof}
\end{prop}

\begin{example}
    $V$ and $I$ are not mutually inverse in general.
    For example, let us set $n=1$ and $K = \mathbb{R}$.
    \begin{itemize}
        \item $IV((x^2)) = I(\{0\}) = (x)$.
        \item $IV((x^2+1)) = I(\emptyset) = \mathbb{R}[x]$.
        \item $VI(\mathbb{R} \setminus \{0\}) = V((0)) = \mathbb{R}$.
    \end{itemize}
\end{example}

\begin{rk}
    The third issue is solved by just restricting the domain of $V$ to the algebraic subsets of $K^n$.
    $VI(s)$ is the smallest algebraic subset of $K^n$ containing $S$ and is called the \emph{Zariski closure} of $S$.
    The first and second issues would be solved by restricting ourselves to \emph{radical} ideals
    and \emph{algebraically closed} fields, respectively.
    The remainder of this section will be spent on showing that.
\end{rk}

\begin{prop}
    $V(S)$ is always a radical ideal.
    \begin{proof}
        Suppose that $f^n \in I(S).$
        Then,
        $\forall (a_1, \ldots, a_n) \in S, 0 = f^n(a_1, \ldots, a_n)
        = f(a_1, \ldots, a_n)^n \Rightarrow f(a_1, \ldots, a_n) = 0$.
        Thus, $f \in I(S)$ and $I(S)$ is radical.
    \end{proof}
\end{prop}

For a moment, we would like to study the maximal ideals of $K$ in relation to the algebraic subsets of $K^n$.

\begin{prop}
    Let $p$ = ($p_1, \ldots, p_n$) be a point in $K^n$.
    Then, the ideal $\mathfrak{m}_p \coloneqq I(\{p\})
    = (T_1 - p_1, \ldots, T_n - p_n) \subset K[T_1, \ldots T_n]$
    is maximal.

    \begin{proof}
        The maximality of $(T_1 - p_1, \ldots, T_n - p_n)$
        comes from the fact that $K[T_1, \ldots T_n]/(T_1 - p_1, \ldots, T_n - p_n) \cong K[p_1, \ldots p_n] = K$
        is a field.
        Now, clearly
        $I(\{p\}) \supseteq (T_1 - p_1, \ldots, T_n - p_n)$
        and
        $1 \notin I(\{p\})$.
        By the maximality of the ideal we found, we must have equality.

    \end{proof}

\end{prop}

\begin{rk}
    However, the converse is not true.
    That is, we can find maximal ideals that are not of the form $\mathfrak{m}_p$.
\end{rk}

\begin{example}
    Let $f$ be any irreducible polynomial in $K[T]$
    of degree $> 1$ (for example, $f = T^2 + 1$ in $\mathbb{R}[T]$).
    Then, $(f)$ is a maximal ideal of $K[T]$ because
    polynomial rings over fields are P.I.D.s.
    Obviously this can't happen in algebraically closed fields.
    We will prove that in fact in algebraically closed fields
    all maximal ideals correspond to points in $K^n$.
\end{example}

\begin{theorem}[Weak Nullstellensatz]
    Any finitely generated $K$-algebra $A$ that is a field is a finite extension of $K$.
    \begin{proof}
        TODO
    \end{proof}
\end{theorem}

\begin{corollary}
    If $K$ is algebraically closed,
    Any finitely generated $K$-algebra $A$ that is a field is $K$.

    \begin{proof} it must be
    generated by algebraic elements over $K$,
    which lie in $K$.
    therefore $A = K$.
    \end{proof}
\end{corollary}

\begin{corollary}
    If $K$ is algebraically closed,
    the only maximal ideals of $K[T_1, \ldots T_n]$ are of the form $\mathfrak{m}_p$.
    \begin{proof}
        Let $\mathfrak{m}$ be a maximal ideal of $K[T_1, \ldots T_n]$.
        Then, $K[T_1, \ldots T_n]/\mathfrak{m}$ is a field so it is just $K$.
        Let $p_i \in K$ be the projection of $T_i$ in $K[T_1, \ldots T_n]/\mathfrak{m}$.
        Then, $\mathfrak{m} \supseteq (T_1 - p_1, \ldots, T_n - p_n) = \mathfrak{m}_p$.
        By maximality of both, we must have equality.
    \end{proof}
\end{corollary}

\begin{theorem}[Hilbert's Nullstellensatz]
    Let $K$ be an algebraically closed field.
    Let $J$ be an ideal of $K[T_1, \ldots T_n]$.
    Then, $I(V(J)) = \text{rad}\,(J)$.

    \begin{proof}
        We will prove this by double inclusion.

        \begin{itemize}
            \item [$\subseteq$] is clear because $IV(J)$ is a radical ideal that contains $J$.
            \item[$\supseteq$]: Let $J = (g_1, \ldots g_m)$ (it is finitely generated by Hilbert's basis theorem). \linebreak
            Let $h \in I(V(J))$.
            We may assume $h \neq 0$.
            Consider the ideal $J' \coloneqq (g_1, \ldots g_m, 1 - hS) \subset K[T_1, \ldots T_{n}, S]$.
            Because $h$ vanishes on $V(J)$, so does $hS$ on $V(J)_{K^{n+1}}$.
            This means that $V(J') = \emptyset$.
            This means that $J' = K[T_1, \ldots T_{n}, S]$.
            Otherwise, $J' \subset \mathfrak{m}_p$ for some maximal ideal $\mathfrak{m}_p$,
            so it would vanish at $p \in K^{n+1}$.
            Therefore, we can write:
            \[
                1 = \sum_{i=1}^m f_i g_i + (1 - hS)f
            \]
            Including in the total ring of fractions, and making the substitution $S = 1/h$,
            and taking common denominators, we get:
            \[
                1 = \frac{1}{h^N} \sum_{i=1}^m \tilde{f_i} g_i \Rightarrow h^N = \sum_{i=1}^m \tilde{f_i} g_i
            \]
            for some $N \in \mathbb{N}$ and $\tilde{f_i} \in K[T_1, \ldots T_n]$.
            That is, $h \in \text{rad}\,(J)$.

        \end{itemize}
    \end{proof}

\end{theorem}


