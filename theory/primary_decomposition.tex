\begin{defn}
        Let $\mathfrak{q} \in A$ be a proper ideal.
        We say that $\mathfrak{q}$ is \emph{primary} if fot all $a, b \in A$
        \[
            ab \in \mathfrak{q}, a \notin \mathfrak{q}
            \Longrightarrow b^n \in \mathfrak{q} \text{ for some } n > 0
        \]
\end{defn}

\begin{rk}
    We have:
    \begin{itemize}
        \item $\mathfrak{p}$ prime ideal $\Longrightarrow \mathfrak{p}$ primary ideal.
        \item $\mathfrak{p}$ primary ideal $\notimplies \mathfrak{p}$ prime ideal.
        \item Let $f: A \longrightarrow B$ be a ring homomorphism.
        Then,
        \[
            \mathfrak{q} \subseteq B \text{ primary ideal }
            \Longrightarrow \mathfrak{q}^c \subseteq A \text{ primary ideal}
        \]
    \end{itemize}
\end{rk}

\begin{prop}
    $\mathfrak{q} \text{ primary ideal } \Longrightarrow rad(\mathfrak{q}) \text{ prime ideal }$
\end{prop}
    \begin{proof}

        Let $ab \in rad(\mathfrak{q})$.
        Suppose $a \notin rad(\mathfrak{q})$.
        In particular, $a \notin \mathfrak{q}$.
        Then, $b^n \in \mathfrak{q}$ for some $n > 0$.
        Therefore, $b \in rad(\mathfrak{q})$.

    \end{proof}

\begin{defn}
    In the conditions of the previous proposition,
    we say that $\mathfrak{q}$
    is $\mathfrak{p}$-primary.
\end{defn}


\begin{prop}
    Let $\mathfrak{q} \subseteq A$ be an ideal s.t.
    $rad(\mathfrak{q}) = \mathfrak{m}$ is a maximal ideal.
    Then $\mathfrak{q}$ is primary.

    \begin{proof}
        Let $ab \in \mathfrak{q}$.
        Consider the inclusions $\mathfrak{m} \subseteq \mathfrak{m} + (b) \subseteq A$.
        Because $\mathfrak{m}$ is maximal, we must have equality in (exactly) one of the two:

        \begin{itemize}

            \item If the second inclusion is an equality,
            \item we have $1 = x + by$, for some $x \in \mathfrak{m}$ and $y \in A$.
            Then, $a = ax + aby \Rightarrow a(1-x) = aby \in \mathfrak{q}$.
            However, reducing modulo $\mathfrak{q}$, we see $\bar{a}\overline{(1-x)} = \bar{0}$
            and $\overline{x} \in \mathcal{N}(A/\mathfrak{q}) \Rightarrow \overline{1-x}$ is a unit.
            Therefore. $\bar{a} = \bar{0}$, which means $a \in \mathfrak{q}$.

            \item If the first inclusion is an equality, we have
            $b \in \mathfrak{m} = rad(\mathfrak{q})
            \Rightarrow b^n \in \mathfrak{q}$ for some $n > 0$.

        \end{itemize}
    \end{proof}

\end{prop}

\begin{defn}
    Let $A$ be a ring and $I \subseteq A$ an ideal.
    A \emph{primary decomposition} of $I$ is a representation of $I$
    as an intersection of primary ideals,
    say, $I = \mathfrak{q}_1 \cap \cdots \cap \mathfrak{q}_n$ with $\mathfrak{q}_i$ primary.
    We say that such a decomposition is \emph{minimal} or \emph{reduced} if:

    \begin{enumerate}
        \item
        $\mathfrak{p}_i \coloneqq \text{rad}(\mathfrak{q}_i)$
        are distinct prime ideals. \label{item:distinct-primes}

        \item
        $\bigcap_{j \neq i} \mathfrak{q}_j \not\subset \mathfrak{q}_i $
        for all $i$. \label{item:noninclusion}
    \end{enumerate}
\end{defn}

\begin{rk}
    Given a primary decomposition of $I$, we can always find a reduced one.
    This is because if the decomposition is not reduced,
    we can decrease the number of primary ideals involved.
    If condition~\ref{item:noninclusion} is not satisfied for some $i$,
    we can remove $\mathfrak{q}_i$ from the decomposition and the equality will still hold.
    If condition~\ref{item:distinct-primes} is not satisfied
    (say, $\mathfrak{p}_i = \mathfrak{p}_j = \mathfrak{p}$),
    then we can replace $\mathfrak{q}_i$ and $\mathfrak{q}_j$ by
    $\mathfrak{q} \coloneqq \mathfrak{q}_i \cap \mathfrak{q}_j$.
    To check that $\mathfrak{q}$ is ($\mathfrak{p}$ -) primary,
    note that (by an extra exercise) $\text{rad}(\mathfrak{q}) = \mathfrak{p}$
    and if $ab \in \mathfrak{q}$ and $a \notin \mathfrak{q}$,
    we can assume $a \notin \mathfrak{q_i} \Rightarrow b \in \mathfrak{p}$.
\end{rk}

\begin{rk}
    However, minimal decompositions are not completely unique.
    In the following propositions, we will produce some invariants of these decompositions.
\end{rk}

\begin{lemma}
    Let $A$ be a ring, let $\mathfrak{q}$ be a $\mathfrak{p}$-primary ideal of $A$
    and let $a \in A$.
    If \linebreak ${(\mathfrak{q} : a) \coloneqq \{x \in A \mid ax \in \mathfrak{q}\}}$,
    then:

    \begin{enumerate}
        \item $a \in \mathfrak{q} \Rightarrow (\mathfrak{q} : a) = A$.
        \item $a \notin \mathfrak{q} \Rightarrow (\mathfrak{q} : a)$ is $\mathfrak{p}$-primary.
        \item $a \notin \mathfrak{p} \Rightarrow (\mathfrak{q} : a) = \mathfrak{q}$
    \end{enumerate}

    \begin{proof}
        We will show them in order:
        \begin{enumerate}

            \item
            Trivial.

            \item
            Let us first prove that $\text{rad}\,((\mathfrak{q} : a)) = \mathfrak{p}$.
            We have inclusions
            $\mathfrak{q} \subseteq (\mathfrak{q} : a) \subseteq \mathfrak{p}$,
            where the second one holds because
            $ab \in \mathfrak{q}, a \notin q \Rightarrow b \in \mathfrak{p}$
            (by definition of $\mathfrak{p}$-primary).
            Taking radicals, we get
            $\mathfrak{p} \subseteq \text{rad}\,((\mathfrak{q} : a)) \subseteq \mathfrak{p}$.
            Now, let $cd = b \in (\mathfrak{q} : a)$ (that is, $acd \in \mathfrak{q}$),
            and suppose $c \notin (\mathfrak{q} : a)$.
            This means that $ac \notin \mathfrak{q}$, so $d \in \mathfrak{p}$.

            \item
            Suppose that $b \in (\mathfrak{q} : a)$ and $b \notin \mathfrak{q}$.
            Because $ab \in \mathfrak{q}$ and $\mathfrak{q}$ is $\mathfrak{p}$-primary,
            this implies $a \in \mathfrak{p}$.

        \end{enumerate}
    \end{proof}
\end{lemma}

\begin{exercise}
    Let $A$ be a ring, $\mathfrak{p}$ a prime ideal of $A$ and $I_1, \ldots, I_n$ ideals of $A$.
    Suppose that $I_1 \cap \cdots \cap I_n \subseteq \mathfrak{p}$.
    Then, there exists $i$ such that $I_i \subseteq \mathfrak{p}$.
    \begin{proof}
        We will proceed by induction on $n$.
        \begin{enumerate}
            \item
            For $n=1$ there is nothing to prove.

            \item
            For $n=2$, suppose that
            $a \in I_1 \setminus \mathfrak{p},
            b \in I_2 \setminus \mathfrak{p}$.
            Then, $ab \in I_1 \cap I_2 \subseteq \mathfrak{p}$,
            which is a contradiction with the fact that $\mathfrak{p}$ is prime.

            \item For $n > 2$, we can apply the $n = 2$ case to
            $I_1 \cap \cdots \cap I_{n-1}$ and $I_n$.
            Then if $I_n \in \mathfrak{p}$ we are done and if
            $I_1 \cap \cdots \cap I_{n-1} \subseteq \mathfrak{p}$,
            we apply the case $(n-1)$st case.
        \end{enumerate}

    \end{proof}
\end{exercise}


\begin{theorem}
    Let $A$ be a ring.
    Let $I = \mathfrak{q}_1 \cap \cdots \cap \mathfrak{q}_n$ be a minimal primary decomposition in $A$.
    Then,
    \[\{\mathfrak{p}_1 \ldots \mathfrak{p}_n\} =
    \{\text{rad}\,(I : a) \mid a \in A\} \cap \text{Spec}\,A\]

    \begin{proof}
        Observe that if $a \in A$,
        \[(I:a) =
        \left( \left(\bigcap_i \mathfrak{q}_i\right) : a \right )
        = \bigcap_i (\mathfrak{q}_i : a)
        \]
        Were the second equality has been shown in an exercise.
        Taking radicals,
        \[
            \text{rad}\,(I:a) = \bigcap_i \text{rad}\,((\mathfrak{q}_i : a))
            = \bigcap_{a \notin \mathfrak{q}_i} \mathfrak{p}_i
        \]
        Where the last equality follows from the previous lemma.
        We are now ready to show both inclusions in the statement:

        \begin{itemize}
            \item[$(\supseteq)$]
            If $\text{rad}\,((I : a))$ is prime,
            by the exercise, there exists $i$ such that
            $\mathfrak{p}_i \subseteq \text{rad}\,(I : a)$.
            But because $\text{rad}\,(I : a)$ is the intersection of them all,
            we must have $\text{rad}\,(I : a) = \mathfrak{p}_i$.

            \item[$(\subseteq)$]
            It is known that they are all prime ideals.
            We need to show that for all $i$, there exists $a \in A$ such that
            $\mathfrak{p}_i = \text{rad}\,(I : a)$.
            It is here that we use the minimality of the decomposition:
            We just choose $a \in \bigcap_{j \neq i} \mathfrak{q}_j \setminus \mathfrak{q}_i $.
            Then, $\text{rad}\,(I : a) = \bigcap_{a \notin \mathfrak{q}_j} \mathfrak{p}_j = \mathfrak{p}_i$.


        \end{itemize}

    \end{proof}
\end{theorem}

\begin{rk}
    This means that the set $\{\mathfrak{p}_1 \ldots \mathfrak{p}_n\}$
    is independent of the choice of the decomposition.
    We call them the \emph{associated primes} of $I$ in $A$
    or $\text{Ass}\,(A/I)$.

\end{rk}

\begin{theorem}
    With the same notation, the set \[\{\mathfrak{q}_i \mid \mathfrak{p}_i \text{ minimal prime of A containing I w.r.t. inclusion}\}\]
    does not depend on the minimal primary decomposition.

    \begin{proof}
        TODO %TODO
    \end{proof}
\end{theorem}

\begin{defn}
    We call this set the \emph{minimal primary components} of $I$ in $A$ or $\text{Min}\,(A/I)$.
\end{defn}

\begin{rk}
    We have now established two senses in which the minimal primary
    decomposition of an ideal is unique,
    but we have not whether it exists in general.
    We will now do so in the Noetherian case.
\end{rk}

\begin{defn}
    Let $A$ be a ring and let $I\subseteq A$ be an ideal.
    We say that $I$ is \emph{irreducible} if,
    for any two ideals $J_1, J_2$ of $A$,
    $I = J_1 \cap J_2 \implies I = J_1 \text{ or } I = J_2$.
\end{defn}

\begin{rk}
    If $I \subseteq A$ is irreducible, then so is $0 \subseteq A / I$. 
\end{rk}

\begin{prop}
    Let $A$ be a Noetherian ring and $I \subseteq A$ an ideal.
    Then, we can write $I = I_1 \cap \cdots \cap I_n$,
    where each $I_i$ is irreducible.

    \begin{proof}
        Consider the set $\mathcal{F}$ of ideals of $A$ that cannot be written in such a way.
        By way of contradiction, suppose that $\mathcal{F}$ is non-empty.
        Let $I \in \mathcal{F}$ be a maximal element.
        By construction, it cannot be written as an intersection of irreducible ideals.
        In particular, $I$ is not irreducible, so we can write $I = J_1 \cap J_2$,
        with $J_1, J_2 \supsetneq I$.
        However, $J_1, J_2 \notin \mathcal{F}$, so we can write
        them as intersections of irreducible ideals:
        $J_1 = I_1 \cap \cdots \cap I_n$
        and
        $J_2 = I_{n+1} \cap \cdots \cap I_{n+m}$.
        Then, $I = I_1 \cap \cdots \cap I_{n+m}$,
        which is a contradiction.

    \end{proof}
\end{prop}

\begin{prop}
    Let $A$ be a Noetherian ring and $I \subseteq A$ an ideal.
    Then,\linebreak
    ${I \text{irreducible} \implies I \text{primary}}$.
    \begin{proof}
        Let $I$ be irreducible and let $ab \in I, a \notin I$.
        Let $I_i \coloneqq (I: b^i)$.
        Clearly, $I_i \subseteq I_{i+1}$
        so we have an ascending chain of ideals, which must stabilize, say at $I_n$.
        Now, set $J_1 \coloneqq I + (a)$ and $ J_2 \coloneqq I + (b^n)$.
        \begin{claim}
            $I = J_1 \cap J_2$.
            \begin{proof}
                The inclusion $I \subseteq J_1 \cap J_2$ is clear.
        To see the other one, let $x \in J_1 \cap J_2$.
        Then, $x = p_1 + ua = p_2 + vb^n $ for some $p_1, p_2 \in I$ and $u, v \in A$.
        Then, $p_1 - p_2 = vb^n - ua \in I$.
        Multiplying by $b$,
        \[
            vb^{n+1} - uab \in I
            \Rightarrow vb^{n+1}\in I
            \Rightarrow v \in I_{n+1} = I_n
            \Rightarrow x = p_2 + vb^n \in I
        \]
            \end{proof}
        \end{claim}
        \noindent By the irreducibility of $I$, we get $I = J_1 = I + (a)$ or $I = J_2 = I + (b)$.
        because we know $a \notin I$, we get $b^n \in I$.
    \end{proof}

\end{prop}

\begin{rk}
    We have just shown that, in a Noetherian ring,
    every ideal has a primary decomposition.
\end{rk}