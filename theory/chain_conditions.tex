\begin{theorem}
    Let A be a ring.
    The following are equivalent:
    \begin{enumerate}

        \item \label{itm:fg}
        Every ideal of $A$ is finitely generated.

        \item \label{itm:acc}
        \emph{\textbf{Ascending chain condition}}:
        every chain of ideals $I_1 \subseteq I_2 \subseteq \cdots \subseteq A$
        is stationary, i.e.
        there exists $n$ such that $I_n = I_{m} = \forall m \geq n$.

        \item \label{itm:max}
        Every non-empty set of ideals of $A$ has a maximal element.

    \end{enumerate}

    \begin{proof}
        We will prove $ \eqref{itm:fg} \implies \eqref{itm:acc} \implies \eqref{itm:max} \implies \eqref{itm:fg}$.
        \begin{itemize}[align=left]

            \item[$\eqref{itm:fg} \Rightarrow \eqref{itm:acc}$:]
            Consider $I \coloneqq \bigcup_{i \in \mathbb{N}} I_n$.
            It is a finitely generated ideal, say $I = \left<a_1, \dots, a_n\right>$.
            Let $a_j \in I_{i_j}$ for each $j \leq n$ and $k = \max_j i_j$.
            Then, $I = I_k$ so the chain is stationary.

            \item[$\eqref{itm:acc} \Rightarrow \eqref{itm:max}$:]
            Let $\mathcal{F}$ be a non-empty set of ideals of $A$.
            Suppose that the condition is false.
            Pick $I_1 \in \mathcal{F}$.
            Because it is not maximal, there exists $I_2 \in \mathcal{F}$ such that $I_1 \subsetneq I_2$.
            By iterating this process, we get a chain $I_1 \subsetneq I_2 \subsetneq \cdots$ which is not stationary.

            \item[$\eqref{itm:max} \Rightarrow \eqref{itm:fg}$:]
            Let $I$ be an ideal of $A$.
            consider the set of finitely generated ideals of $A$ contained in $I$.
            It is non-empty because $0 \in I$.
            By hypothesis, it has a maximal element, say $J$.
            We claim that $J = I$.
            Otherwise,
            there exists $a \in I \setminus J$,
            but then $J + (a)$ is a finitely generated ideal of $A$ contained in
            $I$ which strictly contains $J$.
            in contradiction with the maximality of $J$.
        \end{itemize}
    \end{proof}
\end{theorem}

\begin{defn}
    A ring $A$ satisfying the equivalent conditions of the previous theorem is called \emph{Noetherian}.
\end{defn}

\begin{theorem}
    Let $A$ be a ring.
    The following are equivalent:
    \begin{enumerate}
        \item \label{itm:dcc}
        \emph{\textbf{Descending chain condition}}:
        Every of ideals $A \supseteq I_1 \supseteq I_2 \supseteq \cdots$,
        is stationary, i.e.
        there exists $n$ such that $I_n = I_{m} = \forall m \geq n$.

        \item \label{itm:min}
        Every non-empty set of ideals of $A$ has a minimal element.
    \end{enumerate}

    \begin{proof}
        We will prove $ \eqref{itm:dcc} \implies \eqref{itm:min} \implies \eqref{itm:dcc}$.
        \begin{itemize}[align=left]

            \item[$\eqref{itm:dcc} \Rightarrow \eqref{itm:min}$:]
            Just replace ``maximal'' with ``minimal'' in the proof of $\eqref{itm:acc} \Rightarrow \eqref{itm:max}$.

            \item[$\eqref{itm:min} \Rightarrow \eqref{itm:dcc}$:]
            Let $I_1 \supseteq I_2 \supseteq \cdots$ be a chain of ideals of $A$.
            Then the set $\mathcal{F} \coloneqq \{I_n\}_{n \in \mathbb{N}}$ is non-empty.
            By hypothesis, it has a minimal element, say $I_k$.
            Then, $I_k = I_n$ for all $n \geq k$, each inclusion is shown by the chain condition or by minimality of $I_k$.

        \end{itemize}
    \end{proof}
\end{theorem}

\begin{defn}
    A ring $A$ satisfying the equivalent conditions of the previous theorem is called \emph{Artinian}.
\end{defn}

\begin{rk}
    These two conditions are preserved by taking quotients and rings of fractions.
    This is because we can map all ideals of $A/I$ and $S^{-1} A$ to ideals of $A$
    in an injective way that preserves inclusions (lifting and intersection, respectively).
\end{rk}

\begin{theorem}[Hilbert's basis theorem]
    $A$ Noetherian $\implies A[x]$ Noetherian.
    \begin{proof}
        Suppose that $A$ is Noetherian and let $I \subseteq A[x]$ be an ideal.
        Suppose that it is not finitely generated.
        Let $f_1 \in I \setminus 0,
        f_2 \in I \setminus (f_1),
        f_3 \in I \setminus (f_1, f_2),
        \dots$.
        We may suppose that we have chosen the $f_i$ such that $\deg f_i$ is minimal at each step.
        This means that if the leading term of $f_i$ is $a_i x^{d_i}$,
        $d_1 \leq d_2 \leq d_3 \cdots$.
        Because $A$ is Noetherian, the chain
        $(a_1) \subseteq (a_1, a_2) \subseteq \cdots \subseteq A$
        stabilizes, say, at step k.
        Therefore, we can express, for some $b_i$s $\in A$,
        $a_{k+1} = a_1 b_1 + \cdots + a_{k} b_{k}$.
        Now, consider the polynomial
        \[
            g = f_{k+1} - ( b_1 x^{d_{k+1} - d_1} f_1 + \cdots + b_k x^{d_{k+1} - d_k} f_k)
        \]
        Each of the terms in the sum has degree at most $d_{k+1}$,
        but by construction the degree $k+1$ coefficient of $g$ is zero.
        However, $g \in (f_1, \dots, f_k, f_{k+1}) \setminus (f_1, \dots, f_k)$.
        Otherwise,
        \[
        f_{k+1} = g + b_1 x^{d_{k+1} - d_1} f_1 +
        \cdots + b_k x^{d_{k+1} - d_k} f_k \in (f_1, \dots, f_k)
        \]
        This contradicts the minimality of $d_{k+1}$.

    \end{proof}
\end{theorem}

\begin{rk}
    This in turn implies that $A[x_1, \dots, x_n]$ is Noetherian, by induction on $n$.
\end{rk}

\begin{rk}
    The same is not true for Artinian rings.
    For example, if $\mathbb{K}$ is a field, $\mathbb{K}[x]$ is not Artinian.
    Indeed, the chain $(x) \supsetneq (x^2) \supsetneq (x^3) \supsetneq \cdots$ is not stationary.
\end{rk}